\documentclass[11pt,a4paper,spanish]{book}
\usepackage{biblatex}
\usepackage[utf8]{inputenc}
\addbibresource{}

\begin{document}
	
	\begin{comment}
		En este capítulo se debe indicar el trabajo previo realizado para identificar el problema
		concreto a tratar, así como las posibles soluciones alternativas que se van a evaluar.
		También se deben identificar los criterios de éxito para la comparativa, las medidas que
		se van a tomar, etc.
	\end{comment}
	En este capítulo se identificará el problema en concreto a tratar, a la vez que el diseño de los experimentos para acometerlo. Para ello se exponen los datos utilizados así como un análisis en detalle de estos respondiendo a por qué se escogen esos conjuntos. Finalmente las técnicas de procesamiento y el diseño de la red neuronal propuesta que se usan en este trabajo
	
	El objetivo de esta comparativa es contrastar los resultados obtenidos tras aplicar el mismo sistema de reconocimiento de emociones en la voz entrenado con un lenguaje de referencia, con los otros dos lenguajes escogidos. Mediante esta comparativa se pretende responder a la pregunta si es posible reconocer emociones en un idioma que en principio se desconoce
	
	\subsection{Conjunto de Datos}
	
	
	\subsection{Arquitectura}
	
		\printbibliography
	
\end{document}